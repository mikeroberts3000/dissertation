% \appendix
\section{Appendix}

\subsection{Quadrotor Manipulator Matrices}
\label{sec:ch3:manipulator}

In this subsection, we define the quadrotor manipulator matrices, attempting to be as concise as possible.
We refer the reader to Section \ref{sec:ch2:manipulator_detail} for a more detailed derivation.

We begin by defining the layout of our configuration vector $\mathbf{q}$ as follows,
%
\begin{equation}
\mathbf{q} = 
\begin{bmatrix}
\mathbf{p}_q \\
\mathbf{e}_q
\end{bmatrix}
\end{equation}
%
where
$\mathbf{p}_q$ is the position of the quadrotor;
and $\mathbf{e}_q$ is the vector of Euler angles representing the quadrotor's orientation in the world frame.

We express the manipulator matrices for our quadrotor system as follows,
%
\begin{equation}
\begin{aligned}
\mathbf{H}(\mathbf{q}) = &
\begin{bmatrix}
m\mathbf{I}_{3\times3} & \mathbf{0}_{3\times3} \\
\mathbf{0}_{3\times3}  & \mathbf{I}_{q} \mathbf{R}_{\mathcal{Q},\mathcal{W}} \\
\end{bmatrix} \\
%
\mathbf{C}(\mathbf{q},\dot{\mathbf{q}}) = &
\begin{bmatrix}
\mathbf{0}_{3\times3} & \mathbf{0}_{3\times3}  \\
\mathbf{0}_{3\times3} & \mathbf{I}_q \mathbf{R}_{\mathcal{Q},\mathcal{W}} \dot{\mathbf{A}} - \left(\mathbf{I}_q \mathbf{R}_{\mathcal{Q},\mathcal{W}} \mathbf{A} \dot{\mathbf{e}} \right)_{\times}\mathbf{R}_{\mathcal{Q},\mathcal{W}} \mathbf{A} \\
\end{bmatrix} \\
%
\mathbf{G}(\mathbf{q}) = &
\begin{bmatrix}
-\mathbf{f}_e \\
\mathbf{0}_{3\times1} \\
\end{bmatrix} \\
%
\mathbf{B}(\mathbf{q}) = &
\begin{bmatrix}
\mathbf{R}_{\mathcal{W},\mathcal{Q}} \mathbf{M}_{\mathbf{f}} \\
\mathbf{M}_{\mathbf{\tau}}                                   \\
\end{bmatrix}
\end{aligned}
\end{equation}
%
where $m$ is the mass of the quadrotor;
$\mathbf{I}_q$ is the inertia matrix of the quadrotor;
$\mathbf{R}_{\mathcal{W},\mathcal{Q}}$ is the rotation matrix that represents the quadrotor's orientation in the world frame (i.e., the rotation matrix that maps vectors from the body frame of the quadrotor into the world frame);
$\mathbf{R}_{\mathcal{Q},\mathcal{W}}$ is the rotation matrix that maps vectors from the world frame into the body frame of the quadrotor;
$\mathbf{A}$ is the matrix that relates the quadrotor's Euler angle time derivatives to its angular velocity in the world frame;
$\mathbf{f}_e$ is the external force;
$\mathbf{M}_{\mathbf{f}}$ is the matrix that maps the control input at each of the quadrotor's propellers into a net thrust force oriented along the quadrotor's local $\mathbf{y}$ axis;
$\mathbf{M}_{\mathbf{\tau}}$ is the matrix that maps the control input at each of the quadrotor's propellers into a net torque acting on the quadrotor in the body frame;
$\mathbf{0}_{p \times q}$ is the $p \times q$ zero matrix;
$\mathbf{I}_{k \times k}$ is the $k \times k$ identity matrix;
and the notation $\left( \mathbf{a} \right)_{\times}$ refers to the skew-symmetric matrix, computed as a function of the vector $\mathbf{a}$, such that $\left(\mathbf{a}\right)_{\times}\mathbf{b} = \mathbf{a}\times\mathbf{b}$ for all vectors $\mathbf{b}$.

Our expressions for the manipulator matrices depend on the matrices, $\mathbf{M}_{\mathbf{f}}$ and $\mathbf{M}_{\mathbf{\tau}}$.
We define these matrices as follows, 
%
\begin{equation}
\begin{aligned}
%
\mathbf{M}_{\mathbf{f}} = &
\begin{bmatrix}
0 & 0 & 0 & 0 \\
1 & 1 & 1 & 1 \\
0 & 0 & 0 & 0 \\
\end{bmatrix}\\
%
\mathbf{M}_{\mathbf{\tau}} = &
\begin{bmatrix}
 ds_\alpha & ds_\beta & -ds_\beta & -ds_\alpha \\
\gamma     & -\gamma  & \gamma    & -\gamma    \\
-dc_\alpha & dc_\beta & dc_\beta  & -dc_\alpha \\
\end{bmatrix}
%
\end{aligned}
\end{equation}
%
where $d$, $\alpha$, $\beta$, and $\gamma$ are constants related to the physical design of a quadrotor:
$d$ is the distance from the quadrotor's center of mass to its propellers;
$\alpha$ is the angle in radians that the quadrotor's front propellers form with the quadrotor's positive $\mathbf{x}$ axis;
$\beta$ is the angle in radians that the quadrotor's rear propellers form with the quadrotor's negative $\mathbf{x}$ axis;
$\gamma$ is the magnitude of the in-plane torque generated by the quadrotor propeller producing 1 unit of upward thrust force;
$c_a=\cos a$ and $s_a=\sin a$.

Note that our expressions for the quadrotor manipulator matrices, in particular our expressions for $\mathbf{A}$ and $\mathbf{\dot{A}}$, depend on our choice of Euler angle conventions.
We follow the Euler angle conventions described in Chapter \ref{sec:ch2}.
See the detailed derivation in Section \ref{sec:ch2:manipulator_detail} for details.

\subsection{Setting $v^{\text{\textrm{\textmd{min}}}}$ and $v^{\text{\textrm{\textmd{max}}}}$}
\label{sec:ch3:v_min_v_max}

In our implementation, we set $v^{\text{min}}$ and $v^{\text{max}}$ heuristically, based on the minimum and maximum derivatives we observe in the input progress curve.
In particular, we set $v^{\text{min}}$ and $v^{\text{max}}$ as follows,
%
\begin{equation}
\begin{aligned}
%
v^{\text{min}} & = v^{\text{min}}_{\text{ref}} - \lambda_{\text{proportional}} ( v^{\text{max}}_{\text{ref}} - v^{\text{min}}_{\text{ref}} ) - \lambda_{\text{fixed}} \\
v^{\text{max}} & = v^{\text{max}}_{\text{ref}} + \lambda_{\text{proportional}} ( v^{\text{max}}_{\text{ref}} - v^{\text{min}}_{\text{ref}} ) + \lambda_{\text{fixed}} \\
%
\end{aligned}
\label{eqn:ch3:v_min_v_max}
\end{equation}
%
where $v^{\text{min}}_{\text{ref}}$ and $v^{\text{max}}_{\text{ref}}$ are the minimum and maximum 5$^\text{th}$ time derivatives of the input progress curve;
$\lambda_{\text{proportional}}$ has the effect of padding $v^{\text{min}}$ and $v^{\text{max}}$ proportionally to the range of derivatives observed in input progress curve;
and $\lambda_{\text{fixed}}$ pads $v^{\text{min}}$ and $v^{\text{max}}$ by a fixed amount.
In our implementation, we set $\lambda_{\text{proportional}} = 0.3$ and $\lambda_{\text{fixed}} = 0.001$.
We found that including both proportional and fixed padding terms when setting $v^{\text{min}}$ and $v^{\text{max}}$ improved the overall convergence behavior of our algorithm.
This heuristic assumes that the input progress curve is $C^4$ continuous.
We make this assumption for simplicity, although it could be relaxed by making minor modifications to equation (\ref{eqn:ch3:v_min_v_max}) above.

\subsection{Spacetime Constraints Formulation}
\label{sec:ch3:spacetime}

We begin by concatenating our configuration vector $\mathbf{q}$ and generalized velocity vector $\dot{\mathbf{q}}$ into a single state vector $\mathbf{x}$ as follows,
%
\begin{equation}
\mathbf{x} = 
\begin{bmatrix}
\mathbf{q} \\
\dot{\mathbf{q}}
\end{bmatrix}
\end{equation}
%
We formulate the spacetime constraints optimization problem used in our experiments as follows,
%
\begin{equation*}
\mathbf{X}^{*},\mathbf{U}^{*},\mathbf{T}^* = \argmin_{\mathbf{X},\mathbf{U},\mathbf{T}} \sum_i \left[\lambda_\mathbf{p} \left\Vert \mathbf{p}_i - \mathbf{p}_i^{\text{ref}} \right\Vert_2^2 + \lambda_{dt} \left(dt_i - dt_i^{\text{ref}} \right)^2\right]\\
\end{equation*}
%
\begin{equation}
\begin{aligned}
\text{subject to~~~~}
\mathbf{x}_{0}   & = \mathbf{x}_{0}^{\text{ref}}\\
\mathbf{x}_{N}   & = \mathbf{x}_{N}^{\text{ref}}\\
\mathbf{x}_{i+1} & = \mathbf{x}_{i} +\mathbf{f}(\mathbf{x}_{i},\mathbf{u}_{i})dt_i\\
\mathbf{x}^{\text{min}}  & \leq \mathbf{x}_i  \leq \mathbf{x}^{\text{max}}\\
\mathbf{u}^{\text{min}}  & \leq \mathbf{u}_i  \leq \mathbf{u}^{\text{max}}\\
dt^{\text{min}}          & \leq dt_i          \leq dt^{\text{max}}\\
\end{aligned}
\label{eqn:ch3:spacetime}
\end{equation}
%
where
$\mathbf{X}$ is the concatenated vector of quadrotor states across all time samples;
$\mathbf{U}$ is the concatenated vector of control forces across all time samples;
$\mathbf{T}$ is the concatenated vector of all time deltas;
$\lambda_\mathbf{p}$ and $\lambda_{dt}$ are parameters that trade off the optimizer's preference for matching the spatial layout of the user's input trajectory versus matching the timing of the user's input trajectory;
$\mathbf{p}_i$ is the quadrotor position at time sample $i$;
$\mathbf{p}_i^{\text{ref}}$ is the reference position at time sample $i$ obtained from the user's input trajectory;
$dt_i$ is the time delta from time sample $i$ to time sample $i+1$; 
$dt_i^{\text{ref}}$ is the reference time delta from time sample $i$ to time sample $i+1$ obtained from the user's input trajectory;
$\mathbf{x}_i$ is the quadrotor state at time sample $i$ (note that the state variable $\mathbf{x}_i$ includes the quadrotor's position $\mathbf{p}_i$);
$\mathbf{u}_i$ is the control force vector at time sample $i$;
$\mathbf{f}$ is a function that encodes the quadrotor dynamics;
$\mathbf{x}_0^{\text{ref}}$ and $\mathbf{x}_N^{\text{ref}}$ are the reference start and end states of the quadrotor obtained from the user's input trajectory;
and $\mathbf{x}^{\text{min}}$, $\mathbf{x}^{\text{max}}$, $\mathbf{u}^{\text{min}}$, $\mathbf{u}^{\text{max}}$, $dt^{\text{min}}$, and $dt^{\text{max}}$ are state space limits, control force limits, and time stretching limits imposed on the trajectory.
$\mathbf{X}$, $\mathbf{U}$, and $\mathbf{T}$ are decision variables, everything else is problem data.

Our spacetime constraints formulation depends on the function $\mathbf{f}$, which encodes the quadrotor dynamics.
We define this function in terms of the manipulator matrices from Section \ref{sec:ch3:manipulator} as follows,
%
\begin{equation}
\mathbf{f}(\mathbf{x}_i,\mathbf{u}_i) =
\begin{bmatrix}
\dot{\mathbf{q}}_i  \\
\ddot{\mathbf{q}}_i \\
\end{bmatrix} =
\begin{bmatrix}
\dot{\mathbf{q}}_i  \\
\mathbf{H}^{-1}_i( \mathbf{B}_i\mathbf{u}_i - \mathbf{C}_i\dot{\mathbf{q}}_i - \mathbf{G}_i) \\
\end{bmatrix}
\end{equation}
%
In all our experiments, we set $\lambda_\mathbf{p} = 0.01$ and $\lambda_{dt} = 0.0001$.
We found that these parameter values yielded the best possible computational performance, while still producing trajectories that closely matched the spatial layout the user's input trajectory.

This spacetime constraints formulation  departs from the original formulation by Witkins and Kass \shortcite{witkins:1988}, in the sense that the optimizer is free to stretch time.
This freedom is required in order to ensure that a feasible solution exists.

%Our expressions for the quadrotor camera manipulator matrices also depend on our choice of Euler angle conventions.
%Throughout this paper, we adopt the following Euler angle conventions.
%Let $\mathbf{e} = [ \theta ~ \psi ~ \phi ]^T$ be a vector of Euler angles, and let $\omega$ be angular velocity in the non-rotating frame as these Euler angles change over time. We define the corresponding rotation matrix $\mathbf{R}$ in terms of the Euler angles $\theta$, $\psi$, and $\phi$ as follows,
%%
%\footnotesize
%\begin{equation}
%\begin{aligned}
%%
%\mathbf{R} & = \mathbf{R}_{\mathbf{y}}^{\psi} \mathbf{R}_{\mathbf{z}}^{\theta} \mathbf{R}_{\mathbf{x}}^{\phi} \\
%& = 
%\begin{bmatrix}
%c_\psi  & 0 & s_\psi \\
%0       & 1 & 0 \\
%-s_\psi & 0 & c_\psi \\
%\end{bmatrix}
%\begin{bmatrix}
%1 & 0        & 0 \\
%0 & c_\theta & -s_\theta \\
%0 & s_\theta & c_\theta \\
%\end{bmatrix}
%%
%\begin{bmatrix}
%c_\phi & -s_\phi & 0 \\
%s_\phi & c_\phi  & 0 \\
%0      & 0       & 1 \\
%\end{bmatrix}
%%
%\end{aligned}
%\end{equation}
%\normalsize
%%
%where $c_a=\cos a$ and $s_a=\sin a$.
%Finally, we define the matrix $\mathbf{A}$ that relates $\dot{\mathbf{e}}$ to $\omega$ according to the linear relationship $\mathbf{A} \dot{\mathbf{e}} = \omega$, and its time derivative $\dot{\mathbf{A}}$, as follows,
%%
%\footnotesize
%\begin{equation}
%\begin{aligned}
%%
%\mathbf{A} & =
%\begin{bmatrix}
%c_\psi  & 0 & s_\psi c_\theta \\
%0       & 1 & -s_\theta       \\
%-s_\psi & 0 & c_\psi c_\theta \\
%\end{bmatrix} \\
%%
%\dot{\mathbf{A}} & =
%\begin{bmatrix}
%-s_\psi \dot{\psi} & 0 & -s_\psi s_\theta \dot{\theta} + c_\psi c_\theta \dot{\psi} \\
%0                  & 0 & -c_\theta \dot{\theta} \\
%-c_\psi \dot{\psi} & 0 & -s_\psi c_\theta \dot{\psi} + s_\theta c_\psi \dot{\theta} \\
%\end{bmatrix}
%%
%\end{aligned}
%\end{equation}
%\normalsize
%%
%where $c_a=\cos a$ and $s_a=\sin a$.
