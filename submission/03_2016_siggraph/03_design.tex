\section{Design Principles}
\label{section:design}

In order to design more effective tools for quadrotor camera control, we began by analyzing manuals on cinematography~\cite{mascelli:1965,arijon:1976,katz:1991}, as well as conducting formative interviews. 
We interviewed six professional photographers and videographers.
Their level of expertise with quadrotor cameras ranged from novice to expert.
We accompanied two of the quadrotor experts to professional quadrotor shoots.
All participants primarily fly quadrotors manually, but have used existing trajectory planning tools.
Each interview lasted approximately an hour.
We asked them 30--40 questions pertaining to their setup, their preparations before capture, their workflow during capture, their post-processing steps, and their wish list for quadrotor cinematography.
From this study, we extracted a set of design principles for building effective quadrotor camera planning tools.   

\paragraph{Allow Users to Design Shots Visually}
All participants were primarily concerned with the visual contents of a shot.
For this reason, when flying the quadrotor manually, they relied heavily on a real-time video feed from the camera to decide whether the current shot captures their artistic intent.
Therefore, an effective tool for planning quadrotor camera trajectories should allow users to design shots visually. 

\paragraph{Produce Visually Accurate Shot Previews}
Tools for designing camera trajectories should provide a preview of the entire shot.
This preview needs to be visually accurate. In other words, the frames from the preview shot need to be as visually similar as possible to the real captured frames.
Guaranteeing visual accuracy is challenging, because the physical dynamics of quadrotors impose constraints on the kinds of camera paths that can be executed.
If a shot planning tool does not consider these dynamics when synthesizing camera paths, the quadrotor can deviate significantly from the intended shot during capture, reducing the accuracy of the visual preview.
Therefore, an effective tool should consider the physical dynamics of quadrotors when synthesizing trajectories, in order to create visually accurate shot previews. 

\paragraph{Give Users Precise Timing Control}
Several participants expressed how critical it is to be able to control the timing of a shot.
Indeed, controlling the timing of a shot enables users to specify ease-in and ease-out behavior, which is important in cinematography~\cite{arijon:1976,lasseter:1987}.
Therefore, an effective tool should allow users to precisely control the shot's visual progression over time.

\paragraph{Consider Physical Hardware Limits} 
Quadrotor cameras have inherent physical limits, such as limited maximum thrust, limited maximum velocity, and a limited range of joint angles that are achievable on the camera gimbal.
Attempting to fly a trajectory that does not respect these physical limits can cause the quadrotor to deviate significantly from the intended trajectory, or even crash.
%For example, a user might specify a shot that exceeds the quadrotor's maximum velocity.
%If the user attempted to fly this shot, the quadrotor would lag behind its intended position and cut corners, leading to a shot that is visually different from the user's intended shot.
Indeed, several participants reported destroying equipment in accidents where they misjudged the safety of their camera trajectory or the abilities of their hardware.
Therefore, it is crucial for an effective tool to consider the physical limits of the aircraft.

\paragraph{Provide Users with Spatial Awareness}
Participants often reasoned about the path a camera takes through space.
For example, some participants verbally describe shots by saying \emph{``move from here to there while keeping this in view''} or \emph{``circle around a point''}.
Moreover, users are concerned with the quadrotor's safety around obstacles.
Therefore, an effective tool should provide a virtual environment that is accurately aligned to the real shot location, in order to provide users with meaningful spatial awareness.

\paragraph{Support Rapid Iteration and Provide Repeatability} 
Cinematographers often perform multiple \emph{takes} of the same shot~\cite{mascelli:1965}.
Between takes, they tweak elements of the scene until they achieve their artistic vision.
In support of this workflow, participants expressed the need for tools that support iteration and repeatability with quadrotors. 
In outdoor environments where lighting and weather conditions can change rapidly and greatly affect the quality of a shot, it is important for users to be able to repeat the same shot multiple times. 
In addition, an effective tool should allow users to rapidly iterate, supporting the creative process of exploring and designing shots.

