
% \appendix

\section{System Details}

\subsection{Defining the Quadrotor Camera Manipulator Matrices}
\label{sec:ch2:manipulator}

In this subsection, we define the quadrotor camera manipulator matrices, attempting to be as concise as possible.
We include a detailed derivation of these matrices in the supplementary material.

We begin by defining the layout of our degree-of-freedom vector $\mathbf{q}$, and our control vector $\mathbf{u}$, as follows,
%
\begin{equation}
\mathbf{q} = 
\begin{bmatrix}
\mathbf{p} \\
\mathbf{e}_q \\
\mathbf{e}_g
\end{bmatrix}
%
~~~~
\mathbf{u} = 
\begin{bmatrix}
\mathbf{u}_q \\
\mathbf{u}_g
\end{bmatrix}
\end{equation}
%
where $\mathbf{p}$ is the position of the quadrotor's center of mass; $\mathbf{e}_q$ is the vector of Euler angles representing the quadrotor's orientation in the world frame; $\mathbf{e}_g$ is the vector of Euler angles representing the orientation of the gimbal in the body frame of the quadrotor; $\mathbf{u}_q$ is the thrust control we apply at the quadrotor propellors; and $\mathbf{u}_g$ is the torque control we apply at the gimbal.

We express the manipulator matrices for our quadrotor camera system as follows,
%
\begin{equation}
\begin{aligned}
\mathbf{H}(\mathbf{q}) = &
\begin{bmatrix}
m\mathbf{I}_{3\times3} & \mathbf{0}_{3\times3} & \mathbf{0}_{3\times3} \\
\mathbf{0}_{3\times3}                          & \mathbf{I}_{q} \mathbf{R}_{\mathcal{Q},\mathcal{W}} \mathbf{A}_q & \mathbf{0}_{3\times3} \\
\mathbf{0}_{3\times3}                          & \mathbf{0}_{3\times3} & \mathbf{A}_{g}
\end{bmatrix} \\
%
\mathbf{C}(\mathbf{q},\dot{\mathbf{q}}) = &
\begin{bmatrix}
\mathbf{0}_{3\times3} & \mathbf{0}_{3\times3}  & \mathbf{0}_{3\times3} \\
\mathbf{0}_{3\times3} & \mathbf{I}_q \mathbf{R}_{\mathcal{Q},\mathcal{W}} \dot{\mathbf{A}}_q - \left(\mathbf{I}_q \mathbf{R}_{\mathcal{Q},\mathcal{W}} \mathbf{A}_q \dot{\mathbf{e}}_q \right)_{\times}\mathbf{R}_{\mathcal{Q},\mathcal{W}} \mathbf{A}_q & \mathbf{0}_{3\times3} \\
\mathbf{0}_{3\times3} & \mathbf{0}_{3\times3}  & \dot{\mathbf{A}}_g
\end{bmatrix} \\
%
\mathbf{G}(\mathbf{q}) = &
\begin{bmatrix}
-\mathbf{f}_e \\
\mathbf{0}_{3\times1} \\
\mathbf{0}_{3\times1} \\
\end{bmatrix} \\
%
\mathbf{B}(\mathbf{q}) = &
\begin{bmatrix}
\mathbf{R}_{\mathcal{W},\mathcal{Q}} \mathbf{M}_{\mathbf{f}}    & \mathbf{0}_{3\times3} \\
\mathbf{M}_{\mathbf{\tau}}                                      & \mathbf{0}_{3\times3} \\
\mathbf{0}_{3\times4}                                           & \mathbf{I}_{3\times3} \\
\end{bmatrix}
\end{aligned}
\end{equation}
%
where $m$ is the mass of the quadrotor camera;
$\mathbf{I}_q$ is the inertia matrix of the quadrotor camera;
$\mathbf{R}_{\mathcal{W},\mathcal{Q}}$ is the rotation matrix that represents the quadrotor's orientation in the world frame (i.e., the rotation matrix that maps vectors from the body frame of the quadrotor into the world frame);
$\mathbf{R}_{\mathcal{Q},\mathcal{W}}$ is the rotation matrix that maps vectors from the world frame into the body frame of the quadrotor;
$\mathbf{A}_q$ is the matrix that relates the quadrotor's Euler angle time derivatives to its angular velocity in the world frame;
$\mathbf{A}_g$ is the matrix that relates the gimbal's Euler angle time derivatives to its angular velocity in the body frame of the quadrotor;
$\mathbf{f}_e$ is the external force;
$\mathbf{M}_{\mathbf{f}}$ is the matrix that maps the control input at each of the quadrotor's propellers into a net thrust force oriented along the quadrotor's local $\mathbf{y}$ axis;
$\mathbf{M}_{\mathbf{\tau}}$ is the matrix that maps the control input at each of the quadrotor's propellers into a net torque acting on the quadrotor in the body frame;
$\mathbf{0}_{p \times q}$ is the $p \times q$ zero matrix;
$\mathbf{I}_{k \times k}$ is the $k \times k$ identity matrix;
and the notation $\left( \mathbf{a} \right)_{\times}$ refers to the skew-symmetric matrix, computed as a function of the vector $\mathbf{a}$, such that $\left(\mathbf{a}\right)_{\times}\mathbf{b} = \mathbf{a}\times\mathbf{b}$ for all vectors $\mathbf{b}$.

Our expressions for the quadrotor camera manipulator matrices depend on the matrices, $\mathbf{M}_{\mathbf{f}}$ and $\mathbf{M}_{\mathbf{\tau}}$.
We define these matrices as follows, 
%
\begin{equation}
\begin{aligned}
%
\mathbf{M}_{\mathbf{f}} = &
\begin{bmatrix}
0 & 0 & 0 & 0 \\
1 & 1 & 1 & 1 \\
0 & 0 & 0 & 0 \\
\end{bmatrix} \\
%
\mathbf{M}_{\mathbf{\tau}} = &
\begin{bmatrix}
 ds_\alpha & ds_\beta & -ds_\beta & -ds_\alpha \\
\gamma     & -\gamma  & \gamma    & -\gamma    \\
-dc_\alpha & dc_\beta & dc_\beta  & -dc_\alpha \\
\end{bmatrix}
%
\end{aligned}
\end{equation}
%
where $d$, $\alpha$, $\beta$, and $\gamma$ are constants related to the physical design of a quadrotor:
$d$ is the distance from the quadrotor's center of mass to its propellers;
$\alpha$ is the angle in radians that the quadrotor's front propellers form with the quadrotor's positive $\mathbf{x}$ axis;
$\beta$ is the angle in radians that the quadrotor's rear propellers form with the quadrotor's negative $\mathbf{x}$ axis;
$\gamma$ is the magnitude of the in-plane torque generated by the quadrotor propeller producing 1 unit of upward thrust force;
$c_a=\cos a$ and $s_a=\sin a$.

Note that our expressions for the quadrotor camera manipulator matrices, in particular our expressions for $\mathbf{A}_q$ and $\mathbf{A}_g$, depend on our choice of Euler angle conventions.
See our derivation in Section~\ref{sec:ch2:manipulator_detail} for details.

%Our expressions for the quadrotor camera manipulator matrices also depend on our choice of Euler angle conventions.
%Throughout this paper, we adopt the following Euler angle conventions.
%Let $\mathbf{e} = [ \theta ~ \psi ~ \phi ]^T$ be a vector of Euler angles, and let $\omega$ be angular velocity in the non-rotating frame as these Euler angles change over time. We define the corresponding rotation matrix $\mathbf{R}$ in terms of the Euler angles $\theta$, $\psi$, and $\phi$ as follows,
%%
%\footnotesize
%\begin{equation}
%\begin{aligned}
%%
%\mathbf{R} & = \mathbf{R}_{\mathbf{y}}^{\psi} \mathbf{R}_{\mathbf{z}}^{\theta} \mathbf{R}_{\mathbf{x}}^{\phi} \\
%& = 
%\begin{bmatrix}
%c_\psi  & 0 & s_\psi \\
%0       & 1 & 0 \\
%-s_\psi & 0 & c_\psi \\
%\end{bmatrix}
%\begin{bmatrix}
%1 & 0        & 0 \\
%0 & c_\theta & -s_\theta \\
%0 & s_\theta & c_\theta \\
%\end{bmatrix}
%%
%\begin{bmatrix}
%c_\phi & -s_\phi & 0 \\
%s_\phi & c_\phi  & 0 \\
%0      & 0       & 1 \\
%\end{bmatrix}
%%
%\end{aligned}
%\end{equation}
%\normalsize
%%
%where $c_a=\cos a$ and $s_a=\sin a$.
%Finally, we define the matrix $\mathbf{A}$ that relates $\dot{\mathbf{e}}$ to $\omega$ according to the linear relationship $\mathbf{A} \dot{\mathbf{e}} = \omega$, and its time derivative $\dot{\mathbf{A}}$, as follows,
%%
%\footnotesize
%\begin{equation}
%\begin{aligned}
%%
%\mathbf{A} & =
%\begin{bmatrix}
%c_\psi  & 0 & s_\psi c_\theta \\
%0       & 1 & -s_\theta       \\
%-s_\psi & 0 & c_\psi c_\theta \\
%\end{bmatrix} \\
%%
%\dot{\mathbf{A}} & =
%\begin{bmatrix}
%-s_\psi \dot{\psi} & 0 & -s_\psi s_\theta \dot{\theta} + c_\psi c_\theta \dot{\psi} \\
%0                  & 0 & -c_\theta \dot{\theta} \\
%-c_\psi \dot{\psi} & 0 & -s_\psi c_\theta \dot{\psi} + s_\theta c_\psi \dot{\theta} \\
%\end{bmatrix}
%%
%\end{aligned}
%\end{equation}
%\normalsize
%%
%where $c_a=\cos a$ and $s_a=\sin a$.
