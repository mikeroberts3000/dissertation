\section{Conclusions}

We introduced a set of design principles for quadrotor shot planning tools.
Based on these principles, we built a tool for designing quadrotor camera shots.
Specifically, we added four components to current quadrotor mission planning tools: (1) visual shot design; (2) virtual preview; (3) precise timing control; and (4) visual feasibility feedback. 
Using our tool, both novices and expert users designed compelling shots that would be challenging to create otherwise.
We successfully and autonomously captured all shots with reasonable accuracy on a real quadrotor camera platform.

In the future, we believe quadrotors will enable many creative applications beyond the pre-scripted aerial cinematography shown in this paper.
Quadrotor cameras might soon be able to autonomously film a person skiing down a mountain, or a pack of wild animals hunting prey. 
By flying the same trajectory repeatedly, quadrotor cameras could also enable new kinds of highly dynamic time-lapse video footage.
Moving beyond the domain of cinematography, quadrotor cameras could help to reconstruct virtual 3D models of the physical world with unprecedented coverage and scale.

\section*{Acknowledgements}
We thank the anonymous reviewers for their valuable feedback;
Maxine Lim for her assistance in creating figures, and for designing our project website;
Jane E, James Hegarty, Wilmot Li, and Jerry Talton for their valuable feedback on early drafts of this paper;
Stephen Boyd for the helpful discussion about the optimization problem in Section 7;
Andrew Tridgell, Randy MacKay, and the \textsc{ArduPilot} team for their software support;
\textsc{3D Robotics} for their hardware support;
the professional cinematographers we interviewed, as well as our user study participants, for their time and valuable insights.
This work was supported in part by a NSERC Alexander Graham Bell Canada Graduate Scholarship.
Finally, we dedicate this paper in loving memory of our dear friend and valued collaborator, Floraine Berthouzoz.

%In the near future, we are incorporating a centimeter-accurate GPS into our quadrotor to further improve accuracy.
%We are also experimenting with approaches to algorithmically find feasible trajectories whenever the user creates an infeasible shot.

%More broadly speaking, we believe quadrotor cameras could enable many creative applications beyond the scripted aerial cinematography shown in this paper.
%Quadrotor cameras might soon be able to autonomously film a person skiing down a mountain, or a pack of wild animals hunting prey. 
%By flying the same trajectory repeatedly, quadrotor cameras could also enable new kinds of highly dynamic timelapse video footage.
%Moving beyond the domain of cinematography, quadrotor cameras could help reconstruct virtual 3D models of the physical world with unprecedented coverage and scale.
