\section{Abstract}

% \begin{abstract}
% Drones equipped with cameras are emerging as a powerful tool for large-scale aerial 3D scanning, 
% %but meticulous flight planning and multiple flights are often pre-requisites for obtaining accurate and complete 3D models. 
% but existing automatic flight planners do not exploit all available information about the scene, and can therefore produce inaccurate and incomplete 3D models. 
% We present an automatic method to generate drone trajectories, such that the imagery acquired during the flight will later produce a high-fidelity 3D model. Our method uses a coarse estimate of the scene geometry to plan camera trajectories that: (1) cover the scene as thoroughly as possible; (2) encourage observations of scene geometry from a diverse set of viewing angles; (3) avoid obstacles; and (4) respect a user-specified flight time
% budget. Our method relies on a mathematical model of scene coverage that exhibits an intuitive diminishing returns property known as submodularity.
% We leverage this property extensively to design a trajectory planning algorithm  that reasons globally about the non-additive coverage reward obtained across a trajectory, jointly with the cost of traveling between views.
% We evaluate our method by using it to scan three large outdoor scenes, and we perform a quantitative evaluation using a photorealistic video game simulator.
% \end{abstract}

Drones equipped with cameras are emerging as a powerful tool for large-scale aerial 3D scanning, 
%but meticulous flight planning and multiple flights are often pre-requisites for obtaining accurate and complete 3D models. 
but existing automatic flight planners do not exploit all available information about the scene, and can therefore produce inaccurate and incomplete 3D models. 
We present an automatic method to generate drone trajectories, such that the imagery acquired during the flight will later produce a high-fidelity 3D model. Our method uses a coarse estimate of the scene geometry to plan camera trajectories that: (1) cover the scene as thoroughly as possible; (2) encourage observations of scene geometry from a diverse set of viewing angles; (3) avoid obstacles; and (4) respect a user-specified flight time
budget. Our method relies on a mathematical model of scene coverage that exhibits an intuitive diminishing returns property known as submodularity.
We leverage this property extensively to design a trajectory planning algorithm  that reasons globally about the non-additive coverage reward obtained across a trajectory, jointly with the cost of traveling between views.
We evaluate our method by using it to scan three large outdoor scenes, and we perform a quantitative evaluation using a photorealistic video game simulator.
