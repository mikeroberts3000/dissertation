\vspace{-5pt}
\section{Related Work}

\paragraph{Aerial 3D Scanning and Mapping}
High-quality 3D reconstructions of very large scenes can be obtained using offline multi-view stereo algorithms \cite{furukawa:2015} to process images acquired by drones \cite{pix4d:2015}.
Real-time mapping algorithms for drones have also been proposed, that take as input either RGBD \cite{heng:2011,loianno:2015,michael:2012,sturm:2013} or RGB \cite{wendel:2012} images, and produce as output a 3D\ reconstruction of the scene.
These methods are solving a reconstruction problem, and do not, themselves, generate drone trajectories.
Several commercially available flight planning tools have been developed to assist with 3D scanning \cite{3dr:2017a,pix4d:2017a}.
However, these tools only generate conservative lawnmower and orbit trajectories above the scene.
In contrast, our algorithm generates trajectories that cover the scene as thoroughly as possible, ultimately leading to higher-quality 3D reconstructions.

Generating trajectories that \emph{explore} an unknown environment, while building a map of it, is a classical problem in robotics \cite{thrun:2005}.
Exploration algorithms have been proposed for drones based on local search heuristics \cite{stumberg:2016}, identifying the frontiers between known and unknown parts of the scene \cite{heng:2014,shen:2012}, maximizing newly visible parts of the scene \cite{bircher:2016}, maximizing information gain \cite{charrow:2015b,charrow:2015a}, and imitation learning \cite{choudhury:2017}.
A closely related problem in robotics is generating trajectories that \emph{cover} a known environment \cite{galceran:2013}.
Several coverage path planning algorithms have been proposed for drones \cite{alexis:2015,bircher:2015,heng:2015,hollinger:2013}.
In an especially similar spirit to our work, Heng et al.~propose to reconstruct an unknown environment by executing alternating exploration and coverage trajectories \cite{heng:2015}.
However, existing strategies for exploration and coverage do not explicitly account for the domain-specific requirements of multi-view stereo algorithms (e.g., observing the scene geometry from a diverse set of viewing angles).
Moreover, existing exploration and coverage strategies have not been shown to produce visually pleasing multi-view stereo reconstructions, and are generally not evaluated on multi-view stereo reconstruction tasks.
In contrast, our trajectories cover the scene in a way that explicitly accounts for the requirements of multi-view stereo algorithms, and we evaluate the multi-view stereo reconstruction performance of our algorithm directly.

Several path planning algorithms have been proposed for drones, that explicitly attempt to maximize multi-view stereo reconstruction performance \cite{dunn:2009a,hoppe:2012,mostegel:2016,schmid:2012}.
These algorithms are similar in spirit to ours, but adopt a two phase strategy for generating trajectories.
In the first phase, these algorithms select a sequence of \emph{next-best-views} to visit, ignoring travel costs.
In the second phase, they find an efficient path that connects the previously selected views.
In contrast, our algorithm reasons about these two problems -- selecting good views and routing between them -- jointly in a unified global optimization problem, enabling us to generate more rewarding trajectories, and ultimately higher-quality 3D\ reconstructions.

\vspace{-13pt}
\paragraph{View Selection and Path Planning}
The problem of optimizing the placement (and motion) of sensors to improve performance on a perception task is a classical problem in computer vision and robotics, where it generally goes by the name of \emph{active vision}, e.g., see the comprehensive surveys \cite{chen:2011,scott:2003,tarabanis:1995}.
We discuss directly related work not included in these surveys here.
A variety of active algorithms for 3D scanning with ground-based range scanners have been proposed, that select a sequence of next-best-views \cite{krainin:2011}, and then find an efficient path to connect the views \cite{fan:2016,wu:2014}.
In a similar spirit to our work, Wang et al.~propose a unified optimization problem that selects rewarding views, while softly penalizing travel costs \cite{wang:2007}.
We adapt these ideas to account for the domain-specific requirements of multi-view stereo algorithms, and we impose a hard travel budget constraint, which is an important safety requirement when designing drone trajectories.

Several algorithms have been proposed to select an appropriate subset of views for multi-view stereo reconstruction \cite{dunn:2009b,hornung:2008,mauro:2014b,mauro:2014a}, and to optimize coverage of a scene \cite{ghanem:2015,mavrinac:2013}.
However, these methods do not model travel costs between views.
In contrast, we impose a hard constraint on the travel cost of the path formed by the views we select.

\vspace{-14pt}
\paragraph{Submodular Path Planning}
Submodularity \cite{krause:2014} has been considered in path planning scenarios before, first in the theory community \cite{chekuri:2012,chekuri:2005}, and more recently in the artificial intelligence \cite{singh:2009a,singh:2009b,zhang:2016} and robotics \cite{heng:2015,hollinger:2013} communities.
The coverage path planning formulation of Heng et al.~\cite{heng:2015} is similar to ours, in the sense that both formulations use the same technique for approximating coverage \cite{iyer:2013b,iyer:2013a}.
We extend this formulation to account for the domain-specific requirements of multi-view stereo algorithms, and we evaluate the multi-view stereo reconstruction performance of our algorithm directly.
